%+++++++++++++++++++++++++++++++++++++++++++++++++++++++++++++++++++++++++++++++
%+++ Section "Other Packages"
%+++++++++++++++++++
\section{Other Packages \& Usage}
This section describes all packages still not described above...
% ##############################################################################
\subsection{amsmath}
\subsubsection{Examples}
$$\dbinom{n}{k} = \dbinom{n-1}{k-1} +\dbinom{n-1}{k}$$
\subsubsection{References}

% ##############################################################################
\subsection{babel}
The babel package provides the core architecture for multi-lingual typesetting.
If you are writing in a language other than American English, using this package
is strongly recommended. You should load babel before biblatex and then biblatex
will detect babel automatically.

% ##############################################################################
\subsection{biblatex}
Citing examples:
scrbook \autocite{KOH19}\newline
scrbook \autocite{TEXWELT14}

% ##############################################################################
\subsection{csquotes}
If this package is available, biblatex will use its language sensitive quotation
facilities to enclose certain titles in quotation marks. If not, biblatex uses
quotes suitable for American English as a fallback. When writing in a language
than American English, loading csquotes is strongly recommended.

\subsubsection{Examples}
\subsubsection{References}

% ##############################################################################
\subsection{hyperref}
The package \code{hyperref} is used to handle cross-referencing commands in
LaTeX to produce hypertext links in the document. It must be the last called
package in the preamble. There are of course exceptions (e.g. like the
\code{geometry} package)
\\usepackage[pdfborder={0 0 0}, colorlinks=true, linkcolor=blue]{hyperref}
\subsubsection{Options}
\begin{itemize}
    \item{pdfborder={0 0 0}\\
        No frame around the links}
    \item{colorlinks=true}
    \item{linkcolor=blue}
\end{itemize}
\subsubsection{Examples}
\begin{itemize}
    \item{\textbackslash url\{$<url>$\}:\\
        \url{http://mirrors.ibiblio.org/CTAN/macros/latex/contrib/koma-script/\
        doc/scrguide.pdf\#desc:typearea.option.DIV}}
    \item{\textbackslash href[options]\{$<url>$\}\{$<text>$\}:\\
        \href{http://mirrors.ibiblio.org/CTAN/macros/latex/contrib/koma-script/doc/scrguide.pdf\#desc:typearea.option.DIV}{href link}}
    \item{\textbackslash hyperref: \hyperref{http://mirrors.ibiblio.org/CTAN/macros/latex/contrib/koma-script/doc/scrguide.pdf\#desc:typearea.option.DIV}{category}{name}{text}}
    \item{without: http://mirrors.ibiblio.org/CTAN/macros/latex/contrib/koma-script/doc/scrguide.pdf\#desc:typearea.option.DIV}
\end{itemize}
\subsubsection{References}
https://ctan.org/pkg/hyperref?lang=de
http://ctan.space-pro.be/tex-archive/macros/latex/contrib/hyperref/doc/manual.pdf
https://www.overleaf.com/learn/latex/Hyperlinks
http://www.pirbot.com/mirrors/ctan/macros/latex/contrib/hyperref/doc/manual.html
\begin{itemize}
    \item{}
\end{itemize}

% ##############################################################################
\subsection{url}
\subsubsection{Options}
\begin{itemize}
    \item{hyphens\\
        No frame around the links}
\end{itemize}
% ##############################################################################
\subsection{xcolor}
% ##############################################################################
\subsection{xpatch}
The xpatch package extends the patching commands of etoolbox to biblatex
bibliography macros, drivers and formatting directives.
% ##############################################################################
