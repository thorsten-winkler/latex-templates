\chapter{Document Infos}
% ##############################################################################
%%%DELETE START
The following document infos are only for this template. Using a production
document remove \verb|\usepackage{layouts}| from preamble and this part of the
document, surrounded with \verb+%%%DELETE START+ and \verb+%%%DELETE END+.
The package "layouts" is responsible for a warning "Package layouts Warning:
Layout scale set to 0.5" during compilation.\\

\renewcommand{\pagevalues}{%
%  \begin{center}
Actual page layout values.\\
%\begingroup
%\def\arraystretch{0.7}
%\begin{tabular}{l l}
\twtabular{l l}{
    \texttt{\bs paperheight} = \prntlen{\paperheight}  &
    \texttt{\bs paperwidth}  = \prntlen{\paperwidth}   \\
    \texttt{\bs hoffset}        = \prntlen{\hoffset}        &
    \texttt{\bs voffset}        = \prntlen{\voffset}        \\
    \texttt{\bs evensidemargin} = \prntlen{\evensidemargin} &
    \texttt{\bs oddsidemargin}  = \prntlen{\oddsidemargin}  \\
    \texttt{\bs topmargin}      = \prntlen{\topmargin}      &
    \texttt{\bs headheight}     = \prntlen{\headheight}     \\
    \texttt{\bs headsep}        = \prntlen{\headsep}        &
    \texttt{\bs textheight}     = \prntlen{\textheight}     \\
    \texttt{\bs textwidth}      = \prntlen{\textwidth}      &
    \texttt{\bs footskip}       = \prntlen{\footskip}       \\
    \texttt{\bs marginparsep}   = \prntlen{\marginparsep}   &
    \texttt{\bs marginparpush}  = \prntlen{\marginparpush}  \\
    \texttt{\bs columnsep}      = \prntlen{\columnsep}      &
    \texttt{\bs columnseprule}  = \prntlen{\columnseprule}  \\
    \texttt{\bs marginparwidth}  = \prntlen{\marginparwidth}  \\
    1em = \prntlen{1em}  & 1ex = \prntlen{1ex} \\
}
%\end{tabular}
%\endgroup
%%  \end{center}
}

\printinunitsof{mm}{\pagevalues}\newline

\pagediagram
%%%DELETE END
% ##############################################################################

% ##############################################################################
\section{General Infos \& Usage}
\subsection{Document class}
The document class used in this document is scrbook which belongs to the
KOMA-Script bundle.
\subsubsection{documentclass options}
\begin{singlespace}
\begin{lstlisting}[style=cstm-lists-latex-sty]
\documentclass[
    a4paper,
    appendixprefix=on,
    BCOR=10mm,
    chapterprefix=on,
    DIV=10,
    english,
    fontsize=10pt,
    headsepline=on,
    twoside=off]{scrbook}
\end{lstlisting}
\end{singlespace}
\begin{twitemize}
    \item{BCOR}
    \item{\href{http://mirrors.ibiblio.org/CTAN/macros/latex/contrib/koma-script/doc/scrguide.pdf\#desc:typearea.option.DIV}{DIV:}\\
        DIV infos...}
    \item{headsepline}
    \item{a4paper}
    \item{english}
    \item{twoside}
    \item{fontsize}
    \item{chapterprefix}
    \item{appendixprefix}
\end{twitemize}
\subsubsection{References}
\begin{itemize}
    \item{\href{https://ctan.org/pkg/scrbook}{CTAN - scrbook}}
    \item{\href{http://www.pirbot.com/mirrors/ctan/macros/latex/contrib/koma-script/doc/scrguide.pdf}{Official scrbook document (german)}}
    \item{\href{http://www.pirbot.com/mirrors/ctan/macros/latex/contrib/koma-script/doc/scrguien.pdf}{Official scrbook document (english)}}
\end{itemize}

scrbook \autocite{KOH01}
\dautocite{scrbook}{KOH01}
\subsection{Brackets, Parentheses \& Special Characters}
dsadasdas
https://de.wikibooks.org/wiki/LaTeX-Kompendium:\_Sonderzeichen
\subsection{Fonts}
\subsection{Listings}

\begin{lstlisting}[style=cstm-lists-latex-sty]
test
\end{lstlisting}


\subsection{Line and Page Breaking}
\subsubsection{Examples}
\subsubsection{References}
\begin{itemize}
    \item{\href{http://www.personal.ceu.hu/tex/breaking.htm} {CEU} \autocite{CEU01}}
    \item{\href{https://texwelt.de/wissen/fragen/4014/was-ist-der-unterschied-zwischen-newline-und-linebreak}{Texwelt}}
\end{itemize}


\subsection{Table of Contents}

\section{Packages \& Usage}
% ##############################################################################
\subsection{amsmath}
\subsubsection{Examples}
$$\dbinom{n}{k} = \dbinom{n-1}{k-1} +\dbinom{n-1}{k}$$
\subsubsection{References}

% ##############################################################################
\subsection{babel}
The babel package provides the core architecture for multi-lingual typesetting.
If you are writing in a language other than American English, using this package
is strongly recommended. You should load babel before biblatex and then biblatex
will detect babel automatically.

% ##############################################################################
\subsection{biblatex}
Citing examples:
scrbook \autocite{KOH01}\newline
scrbook \autocite{SOR01}\newline
scrbook \autocite{GER01}\newline
scrbook \autocite{GER02}\newline
scrbook \autocite{LOS03}\newline
scrbook \autocite{BAL01}\newline
scrbook \autocite{BAL02}\newline

% ##############################################################################
\subsection{csquotes}
If this package is available, biblatex will use its language sensitive quotation
facilities to enclose certain titles in quotation marks. If not, biblatex uses
quotes suitable for American English as a fallback. When writing in a language
than American English, loading csquotes is strongly recommended.

\subsubsection{Examples}
\subsubsection{References}

% ##############################################################################
\subsection{hyperref}
The package \enquote{hyperref} is used to handle cross-referencing commands in
LaTeX to produce hypertext links in the document. It must be the last called
package in the preamble. There are of course exceptions (e.g. like the
\enquote{geometry} package)
\\usepackage[pdfborder={0 0 0}, colorlinks=true, linkcolor=blue]{hyperref}
\subsubsection{Options}
\begin{itemize}
    \item{pdfborder={0 0 0}\\
        No frame around the links}
    \item{colorlinks=true}
    \item{linkcolor=blue}
\end{itemize}
\subsubsection{Examples}
\begin{itemize}
    \item{\textbackslash url\{$<url>$\}:\\
        \url{http://mirrors.ibiblio.org/CTAN/macros/latex/contrib/koma-script/\
        doc/scrguide.pdf\#desc:typearea.option.DIV}}
    \item{\textbackslash href[options]\{$<url>$\}\{$<text>$\}:\\
        \href{http://mirrors.ibiblio.org/CTAN/macros/latex/contrib/koma-script/doc/scrguide.pdf\#desc:typearea.option.DIV}{href link}}
    \item{\textbackslash hyperref: \hyperref{http://mirrors.ibiblio.org/CTAN/macros/latex/contrib/koma-script/doc/scrguide.pdf\#desc:typearea.option.DIV}{category}{name}{text}}
    \item{without: http://mirrors.ibiblio.org/CTAN/macros/latex/contrib/koma-script/doc/scrguide.pdf\#desc:typearea.option.DIV}
\end{itemize}
\subsubsection{References}
https://ctan.org/pkg/hyperref?lang=de
http://ctan.space-pro.be/tex-archive/macros/latex/contrib/hyperref/doc/manual.pdf
https://www.overleaf.com/learn/latex/Hyperlinks
http://www.pirbot.com/mirrors/ctan/macros/latex/contrib/hyperref/doc/manual.html
\begin{itemize}
    \item{}
\end{itemize}
% ##############################################################################
\subsection{inputenc}
% ##############################################################################
\subsection{listings}
% ##############################################################################
\subsection{url}
\subsubsection{Options}
\begin{itemize}
    \item{hyphens\\
        No frame around the links}
\end{itemize}
% ##############################################################################
\subsection{xcolor}
% ##############################################################################
\subsection{xpatch}
The xpatch package extends the patching commands of etoolbox to biblatex
bibliography macros, drivers and formatting directives.

% ##############################################################################
